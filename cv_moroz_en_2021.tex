%%%%%%%%%%%%%%%%%%%%%%%%%%%%%%%%%%%%%%%%%
% Medium Length Professional CV
% LaTeX Template
% Version 2.0 (8/5/13)
%
% This template has been downloaded from:
% http://www.LaTeXTemplates.com
%
% Original author:
% Trey Hunner (http://www.treyhunner.com/)
%
% Important note:
% This template requires the resume.cls file to be in the same directory as the
% .tex file. The resume.cls file provides the resume style used for structuring the
% document.
%
%%%%%%%%%%%%%%%%%%%%%%%%%%%%%%%%%%%%%%%%%

%----------------------------------------------------------------------------------------
%	PACKAGES AND OTHER DOCUMENT CONFIGURATIONS
%----------------------------------------------------------------------------------------

\documentclass{resume} % Use the custom resume.cls style
\usepackage[english,russian]{babel}   %% загружает пакет многоязыковой вёрстки
\usepackage{fontspec}      %% подготавливает загрузку шрифтов Open Type, True Type и др.
\defaultfontfeatures{Ligatures={TeX},Renderer=Basic}  %% свойства шрифтов по умолчанию
\setmainfont[Ligatures={TeX,Historic},
SmallCapsFont={Brill},
SmallCapsFeatures={Letters=SmallCaps}]{Brill} %% задаёт основной шрифт документа
\setsansfont{Brill}                    %% задаёт шрифт без засечек
\setmonofont{Iosevka}
\usepackage{indentfirst}


\usepackage[left=0.75in,top=0.6in,right=0.75in,bottom=0.6in]{geometry} % Document margins

\name{George Moroz} % Your name
\address{agricolamz@gmail.com, agricolamz@hse.ru}
\address{https://github.com/agricolamz/}

\begin{document}

%----------------------------------------------------------------------------------------
%	EDUCATION SECTION
%----------------------------------------------------------------------------------------

\begin{rSection}{Education}

{\bf HSE University} \hfill {\em 2021} \\ 
Defended PhD on ``Some questions of Circassian segmental and suprasegmental phonology and phonetics''\\
{\bf Lomonosov Moscow State University} \hfill {\em 2012--2015} \\ 
doctoral studies (PhD)\\
Research on \textit{Phonology of Circassian languages}\\
{\bf Russian State University for Humanities, Moscow} \hfill {\em 2007--2012} \\ 
Specialist in Linguistics

{\bf Additional:}
\begin{itemize}
\item {\bf The incumbent member of Literary Translation Seminar ``Transatlantic''} \hfill {\em 2011--2017}
\item {\bf Centre of Polish Language and Culture University of Warsaw} \hfill {\em August 2010}
\item {\bf School of Polish Language and Culture University of Silesia} \hfill {\em August 2009}
\end{itemize}

\end{rSection}

%----------------------------------------------------------------------------------------
%	WORK EXPERIENCE SECTION
%----------------------------------------------------------------------------------------

\begin{rSection}{Lecturing and research}
\begin{rSubsection}{Linguistic Convergence Laboratory (NRU HSE)}{February 2018 --- present}{junior research fellow}{Moscow}
\item Data Analysis, Phonetics, Linguistic maps, Caucasian languages
\end{rSubsection}
\begin{rSubsection}{National Research University Higher School of Economics}{September 2013 --- present}{lecturer, senior lecturer, associate professor}{Moscow}
\item Statistics, Data Analysis, R, \LaTeX, ELAN, Phonetics and Phonology, Polish, Writing systems et alia
\end{rSubsection}
\begin{rSubsection}{Summer school on Data Analysis}{August 2017 --- August 2020}{lecturer, co-director}{Tver Oblast}
\item R, Automatic Text Analysis, Frequentist inference, Bayesian inference, Phonetic Data Analysis, Shiny, Rmarkdown
\end{rSubsection}
\begin{rSubsection}{Laboratory of the Languages of the Caucasus (NRU HSE)}{January 2015 --- January 2018}{junior research fellow}{Moscow}
\item Phonetics, Phonology, Linguistic maps, Caucasian languages
\end{rSubsection}
\begin{rSubsection}{Russian State University for Humanities}{September 2014 --- January 2016}{lecturer}{Moscow}
\item Phonetics and Phonology
\end{rSubsection}
\begin{rSubsection}{Russian State University for Humanities}{February 2014 --- May 2014}{lecturer}{Moscow}
\item Polish
\end{rSubsection}
%------------------------------------------------

\end{rSection}

\begin{rSection}{Grants}
\begin{rSubsection}{FFLI [Foundation for Fundamental Linguistic Research]}{2011 --- 2012}{researcher}{}
\item research grant No. A-46 to Nina Sumbatova (primary researcher): Grammatical description of Dargwa dialects
\end{rSubsection}
\begin{rSubsection}{RFFI [Russian Foundation for Basic Research]}{2015 --- 2017}{researcher}{}
\item research grant No. 15-06-07434a to Yury Lander (primary researcher): Computer-based documentation of a polysynthetic language
\end{rSubsection}
\begin{rSubsection}{RFFI [Russian Foundation for Basic Research]}{2018 --- 2020}{researcher}{}
\item research grant No. 18-012-00852 to Michael Daniel (primary researcher): The Andi morphosyntax in a typological perspective
\end{rSubsection}
\end{rSection}

\begin{rSection}{Linguistic Fieldtrips}

\begin{rSubsection}{Circassian languages --- 11}{2010, 2011, 2012, 2013, 2014, 2015, 2016, 2017, 2018}{}{}
\item The Republic of Adygea, Krasnodar Krai, the Karachay-Cherkess Republic, the Kabardino-Balkar Republic
\end{rSubsection}
\begin{rSubsection}{Dargwa and Andi --- 11}{2011, 2012, 2013, 2014, 2015, 2016, 2017, 2018, 2019}{}{}
\item The Republic of Dagestan
\end{rSubsection}
\begin{rSubsection}{Abaza --- 4}{2017, 2018, 2019, 2021}{}{}
\item The Karachay-Cherkess Republic
\end{rSubsection}
\begin{rSubsection}{Nanai language --- 1}{2013}{}{}
\item Khabarovsk Krai
\end{rSubsection}
\end{rSection}

%----------------------------------------------------------------------------------------
%	TECHNICAL STRENGTHS SECTION
%----------------------------------------------------------------------------------------

\begin{rSection}{Technical Strengths}

\begin{tabular}{ @{} >{\bfseries}l @{\hspace{6ex}} l }
Programming & R (main language), Praat, Git, GitHub Classroom, html, css, bash\\
Geographic Information Systems & Google Earth, Google Maps, R packages\\
Language analysis & Praat, ELAN, EXMARaLDA, Audacity, FieldWorks, SplitsTree, R packages\\
Document Creation & \LaTeX, Beamer, RMarkdown, Office\\
\end{tabular}

\end{rSection}

\begin{rSection}{Packages}
\begin{tabular}{ @{} >{\bfseries}l @{\hspace{6ex}} l }
lingtypology  & Moroz G. (2017). lingtypology: easy mapping for Linguistic Typology. \\
& <URL: https://CRAN.R-project.org/package=lingtypology>.\\
phonfieldworks & Moroz G. (2019). Phonetic fieldwork and experiments with phonfieldwork
package.\\
& <URL: https://CRAN.R-project.org/package=phonfieldwork>.\\
checkdown & Moroz G. (2020). Create check-fields and check-boxes with checkdown\\
& <URL: https://CRAN.R-project.org/package=checkdown>.\\
lingglosses & Moroz, G. (2021) lingglosses: Linguistic glosses and semi-automatic list of glosses creation.\\
& <URL: https://CRAN.R-project.org/package=lingglosses>.\\
\end{tabular}

\end{rSection}

\begin{rSection}{Linguistic Databases}
\begin{itemize}
\item The Circassian Consonants Correspondences Database (CCCD) \\ <URL: https://agricolamz.github.io/cccd/> 
\item Iconicity patterns in Sign Languges (IPSL) \\ <URL: https://sl-iconicity.shinyapps.io/iconicity\_patterns/>
\item The World Consonant Alternation Database (WCAD) \\ <URL: https://agricolamz.github.io/wcad/>
\item The World Writing System Database (WWSD) \\ <URL: https://agricolamz.github.io/wwsd/>
\item Daghestanian Sound Database \\ <URL: https://daghestanian-sound-database.herokuapp.com/>
\item Лексика адыгских идиомов на территории РФ <URL: https://agricolamz.github.io/adyghe\_atlas/>
\item The Circassian Consonants Correspondences Database (CCCD) \\ <URL: https://agricolamz.github.io/cccd/>
\item Non semantically motivated Noun Classes \\ <URL: https://agricolamz.github.io/non\_semantically\_motivated\_noun\_classes\_db/> \vspace{0.6cm}
\end{itemize}
\end{rSection}

\begin{rSection}{Domains}
field linguistics, language documentation, phonetics, phonology, writing systems, linguistic geography, languages of the Caucasus, statistics, computer instruments for linguistic research, low-resource NLP
\end{rSection}

\begin{rSection}{Created online courses}
\begin{itemize}
\item R для лингвистов: программирование и анализ данных\\
<https://openedu.ru/course/hse/RLING/>
\item Инструментальная фонетика: компьютерные методы сбора и анализа данных\\
<https://www.coursera.org/learn/instrumental-phonetics>
\end{itemize}

\end{rSection}

\begin{rSection}{Hackathons}
\begin{itemize}
\item 2018.10.06  SpendingSprint by ClearSpending  --- \textbf{II place}
\item 2018.10.20--21  MosFinData by ClearSpending and Moscow department of finance
 --- \textbf{Special thanks} 
\item  2019.04.27--28 II Hackathon by Novaya gazeta --- \textbf{III place} 
\item 2019.12.13--15 I online hackathon Devs against the machine --- \textbf{II place} 
\end{itemize}

\end{rSection}


\begin{rSection}{Publications}
\begin{enumerate}
\item Мороз Г. А. Замечания о морфологии числительных в адыгейском языке // В кн.: Полевые исследования студентов РГГУ: Этнология, фольклористика, лингвистика, религиоведение. Вып. VI. М. : РГГУ, 2011. С. 230--245.
\item Мороз Г. А. Адыгский, адыгейский, бесленеевский, уляпский: представления носителей уляпского говора о своем языке // В кн.: «Народная лингвистика»: взгляд носителей языка на язык. Тезисы докладов международной научной конференции / Отв. ред.: Е. Головко. СПб. : Нестор-История, 2012. С. 42--44.
\item Мороз Г. А. Консонантная система уляпского говора в сопоставлении с аналогами других диалектов адыгских языков // В кн.: Полевые исследования студентов РГГУ: Этнология, фольклористика, лингвистика, религиоведение. Вып. VII / Под общ. ред.: Л. Г. Жукова, О. А. Казакевич, В. Л. Кляус, Н. А. Козлова, Ю. А. Ландер, А. Б. Мороз, Ю. В. Филиппов. М. : РГГУ, 2012. С. 193--200.
\item Мороз Г. А. Ударение в уляпском говоре кабардинского языка // В кн.: Проблемы языка: Сборник научных статей по материалам Первой конференции-школы «Проблемы языка: взгляд молодых ученых» (20-22 сентября 2012 г.). М. : Институт языкознания РАН, 2012. С. 190--204.
\item Мороз Г. А. Фонология и фонетика консонантной системы мегебского языка в сопоставлении с другими даргинскими идиомами // В кн.: Сборник научных статей по материалам Второй конференции-школы «Проблемы языка: взгляд молодых ученых». М. : Институт языкознания РАН, 2013. С. 277--288. (в печати)
\item Moroz G. Stress in Mehweb: A Lexically Filled Optimality Theory Approach / NRU HSE. Series WP BRP "Linguistics". 2014. No. WP BRP 19/LNG/2015.
\item Мороз Г. А. Именное ударение в даргинских языках // В кн.: Актуальные вопросы теоретической и прикладной фонетики. Сборник статей к юбилею О.Ф. Кривновой. М. : БукиВеди, 2014. С. 245--269.
\item Мороз Г. А. Особенности систем числительных языков Кавказа // В кн.: Логический анализ языка: Числовой код в разных языках и культурах / Отв. ред.: Н. Д. Арутюнова; под общ. ред.: Н. Д. Арутюнова; науч. ред.: Н. Д. Арутюнова, М. Л. Ковшова, С. Ю. Бочавер. М. : УРСС, 2014. С. 171--182.
\item Мороз Г. А. Структура составных числительных языков кавказа в типологической перспективе: сложение // Вестник Московского государственного гуманитарного университета им. М.А. Шолохова. Филологические науки. 2014. № 1. С. 85--100.
\item Мороз Г. А. Адыгские идиомы в Турции: от первых описаний до собственной письменности // Вестник РГГУ. Серия «Филологические науки. Языкознание»/ Московский лингвистический журнал. 2015. С. 44--60.
\item Кюсева М. В., Мороз Г. А. Лексика со значением формы и размера в Русском Жестовом Языке // В кн.: Специальные образовательные условия и качество профессиональной подготовки лиц с ограниченными возможностями здоровья. Новосибирский государственный технический университет, 2015. С. 120--127.
\item Ландер Ю. А., Аркадьев П. М., Мороз Г. А. Об исследовании бжедугского диалекта адыгейского языка // В кн.: Полевые исследования студентов РГГУ: Этнология, фольклористика, лингвистика. Вып. X / Под общ. ред.: П. М. Аркадьев, О. А. Казакевич, В. Л. Кляус, Н. А. Козлова, Ю. А. Ландер, Е. В. Левкиевская, А. Б. Мороз, Ю. В. Филиппов. М. : РГГУ, 2015. С. 183--201.
\item Багирокова И. Г., Ландер Ю. А., Мороз Г. А. Понятие основы и морфологический анализ адыгейской словоформы // В кн.: IV Международный симпозиум лингвистов-кавказоведов. Вопросы структуры основы и корня иберийско-кавказских языков. Тб. : [б.и.], 2015. С. 107--110.
\item Мороз Г. А. Адвербиальные конструкции временной дистрибуции в балто-славянских языках: ареальное и корпусное исследование // В кн.: Acta Linguistica Petropolitana. Труды Института лингвистических исследований РАН. (Том XII, часть 1) / Отв. ред.: Н. Казанский. Т. XII. Ч. 1. СПб. : Наука, 2016. С. 151--167.
\item Мороз Г. А. Cкорости речи носителей кубанского диалекта кабардино-черкесского языка: устный дискурс vs. чтения текста // Томский журнал лингвистических и антропологических исследований. 2017. №  2. С. 9--17.
\item Moroz G., Martynova A. Uvular consonants in Languages of the Caucasus, in: Historical Linguistics of the Caucasus: Book of abstracts. Paris, 12-14 April, 2017 / Историческое изучение языков Кавказа. Тезисы докладов Международной научной конференции. Париж, 12--14 апреля 2017 г. / Сост.: Т. А. Майсак. Махачкала : ИЯЛИ ДНЦ РАН, 2017. P. 150--153.
\item Мороз Г. А., Зибер И. А. К типологии систем сибилянтов: параметры вариативности [s] // В кн.: Четырнадцатая конференция по типологии и грамматике для молодых исследователей. Тезисы докладов. (Санкт-Петербург, 23--26 ноября 2017 г.). СПб. : Издательство Нестор-История, 2017. С. 113--115.
\item Багирокова И. Г., Ландер Ю. А., Мороз Г. А. О выражении множественных непрямых объектов в адыгейском глаголе // В кн.: Становление и развитие младописьменных языков. К 120-летию со дня рождения выдающегося языковеда, основоположника адыгейского языкознания Д.А. Ашхамафа: материалы Международной научной конференции (Майкоп, 21--23 июня 2017 г.). Майкоп : [б.и.], 2017. С. 23--27.
\item  Мороз Г. А., Романова К. И. 'Искать' и 'найти' в абазинском языке // В кн.: ЕВРика! Сборник статей о поисках и находках к юбилею Е.В. Рахилиной / Под общ. ред.: Д. А. Рыжова, Н. Р. Добрушина, А. А. Бонч-Осмоловская, А. С. Выренкова, М. В. Кюсева, Б. В. Орехов, Т. И. Резникова. М. : Лабиринт, 2018. С. 100--104.
\item Moroz G., Plaskovitskaya A. A., Rudnev P. Automatic detection of natural phonological classes in Russian Sign Language / NRU HSE. Series WP BRP "Linguistics". 2018. No. 74.
\item Kimmelman V., Klezovich A., Moroz G. IPSL: A Database of Iconicity Patterns in Sign Languages. Creation and Use, in: Proceedings of the Eleventh International Conference on Language Resources and Evaluation (LREC 2018). P. : European Language Resources Association (ELRA), 2018. Ch. 102. P. 4230--4234.
\item Verhees S., Moroz G. Time and time again: The evolution of ‘time’-nouns into temporal clause markers in three Daghestanian languages, in: Пятнадцатая Конференция по типологии и грамматике для молодых исследователей. Тезисы докладов (Санкт-Петербург, 22-24 ноября 2018 г.). СПб. : ИЛИ РАН, 2018. P. 212--214.
\item Мороз Г. А. Возможен ли отказ от выборки? Байесовский подход к типологическому исследованию // В кн.: Пятнадцатая Конференция по типологии и грамматике для молодых исследователей. Тезисы докладов (Санкт-Петербург, 22-24 ноября 2018 г.). СПб. : ИЛИ РАН, 2018. С. 91--94.
\item Мороз Г. А., Ферхеес Я. Х. Классы в Зило: экспериментальные данные андийского языка // В кн.: Малые языки в большой лингвистике. Сборник трудов конференции 2017. М. : Буки Веди, 2018. Гл. 22. С. 145--151.
\item Мороз Г. А. Маркирование актантов многоместных предикатов в польском языке // В кн.: Валентностные классы глаголов и их вариативность в разноструктурных языках / Отв. ред.: С. С. Сай. СПб. : Наука, 2018. С. 183--197.
\item Аркадьев П. М., Ландер Ю. А., Летучий А. Б., Мороз Г. А., Тестелец Я. Г. Типологически ориентированное описание абазинского языка: проспект исследований // Вестник Карачаево-Черкесского государственного университета. 2018. № 44. С. 95--100.
\item  Dobrushina N., Kozhukhar A. A., Moroz G. Gendered multilingualism in highland Daghestan: story of a loss // Journal of Multilingual and Multicultural Development. 2019. Vol. 40. No. 2. P. 115--132.
\item Мороз Г. А. Слоговая структура адыгейского языка // Вопросы языкознания. 2019. № 2. С. 82--95.
\item  Зибер И. А., Мороз Г. А. Исследование акустической вариативности s методом главных компонент // Вестник Новосибирского государственного университета. Серия: Лингвистика и межкультурная коммуникация. 2019. № 1. С. 49--64.
\item Moroz George, Verhees Samira Variability in noun classes assignment in Zilo Andi: experimental data // Iran and the Caucasus. 2019. V. 23. № 3. pp. 268--282.
\item  Moroz G. Phonology of Mehweb, in: The Mehweb language: Essays on phonology, morphology and syntax / Ed. by M. Daniel, N. Dobrushina, D. Ganenkov. Berlin : Language Science Press, 2019. doi Ch. 3. P. 22--34.
\item Arkhipov A., Daniel M., Belyaev O., Moroz G., Esling J. H. A reinterpretation of lower-vocal-tract articulations in Caucasian languages, in: Proceedings of the 19th International Congress of Phonetic Sciences, Melbourne, Australia 2019. Australasian Speech Science and Technology Association Inc., 2019. P. 1550--1554.
\item Mamonova T., Moroz G. VOT features of consonants in Abaza / NRU HSE. Series WP BRP "Linguistics". 2019. No. 91/LNG/2019.
\item Рошан Н., Мороз Г. А. Семантика некаузирирующих каузативов в андийском языке // В кн.: Языки и культуры народов России и мира. Махачкала : АЛЕФ, 2019. С. 348-352.
\item Moroz G. Length of East Caucasian subject indexes: a quantative research, in: Дурхъаси хазна. Сборник статей к 60-летию Р. О. Муталова / Под общ. ред.: Т. А. Майсак, Н. Р. Сумбатова, Я. Г. Тестелец. М. : Буки Веди, 2021. P. 258-282.
\item Dobrushina N., Moroz G. The speakers of minority languages are more multilingual // International Journal of Bilingualism. 2021, Vol. 25. No. 4. P. 921-938. https://doi.org/10.1177/13670069211023150
\item Seržant I. A., Moroz G. Universal attractors in language evolution provide evidence for the kinds of efficiency pressures involved // Humanities and Social Sciences Communications. 2022.

\end{enumerate}
\end{rSection}

\begin{rSection}{Science popularization}
\begin{enumerate}
\item Георгий Мороз (2018)  Ни Yanny, ни Laurel, \\ nplus1.ru/blog/2018/05/16/acousticdress
\item Сергей Кузнецов, Тарас Молотилин, Георгий Мороз (2019) Картография и хронология жадин, \\ nplus1.ru/material/2019/06/19/greedy
\item Кристина Сойкина, Георгий Мороз (2019) Салат для большой страны, \\ nplus1.ru/material/2019/10/17/olivier-great-recipe
\end{enumerate}
\end{rSection}

\begin{rSection}{Translations}
\begin{enumerate}
\item Филипович К. Да не знает левая рука твоя... // В кн.: Тени. М.: Знак, 2014. С.~43--62, ISBN: 978--5--7516--1228--3 (K. Filipowicz (1960) Niech nie wie lewica..., z książki „Biały ptak”, 48--65)
\item Колаковский Л. Война с вещами. // В кн.: Сказки из королевства Лайлонии для больших и маленьких. СПб.: Jaromir Hladik press. C.~94--103, ISBN: 978-5-6046601-5-7 (L. Kołakowski Wojna s rzeczami, ze zbioru 13 bajek z królestwa Lailonii dla dużych i małych)
\end{enumerate}
\end{rSection}

%----------------------------------------------------------------------------------------
%	EXAMPLE SECTION
%----------------------------------------------------------------------------------------

%\begin{rSection}{Section Name}

%Section content\ldots

%\end{rSection}

%----------------------------------------------------------------------------------------

\end{document}
